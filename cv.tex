\documentclass[a4paper]{article}
\usepackage[utf8x]{inputenc}
\usepackage[english]{babel}
\usepackage[calc,showdow,english]{datetime2}
\usepackage[margin=1cm]{geometry}
\usepackage{helvet}
\renewcommand{\familydefault}{\sfdefault}
\usepackage{tikz}
\usepackage{graphicx}
\usepackage{enumitem}
\usepackage{xcolor}
\usepackage{textcomp}
\usetikzlibrary{calc}
\graphicspath{{icons/}}
\def\newicon#1{%
  \tikz[baseline=0.1em]{
    \path[use as bounding box] (0, 0) rectangle (1em, 1em);
    #1
  }%
}
\def\newsvgicon#1#2{
  \def\svgwidth{1em}%
  \tikz[baseline=-#2]{\node[inner sep=0em, use as bounding box, text width=1em]{\input{#1}}}%
}
\def\newsvgiconsize#1#2#3{
  \def\svgwidth{#3}%
  \tikz[baseline=-#2]{\node[inner sep=0em, use as bounding box, text width=#3]{\input{#1}}}%
}

\def\ucsbicon{\newsvgicon{icons/UCSB.pdf_tex}{0.95em}}
\def\ucsbwordicon{\newsvgiconsize{icons/ucsb-word.pdf_tex}{0.9em}{3em}}
\def\bilibiliicon{\newsvgicon{icons/bilibili.pdf_tex}{0.4em}}
\def\bluebicon{\newsvgicon{icons/bilibiliblue.pdf_tex}{0.4em}}
\def\bilibiliwordicon{\newsvgiconsize{icons/bilibili-word.pdf_tex}{0.4em}{2.3em}}
\def\githubicon{\newsvgicon{icons/github.pdf_tex}{0.85em}}
\def\ghwhiteicon{\newsvgicon{icons/githubwhite.pdf_tex}{0.4em}}
\def\blueghicon{\hskip 0em\newsvgicon{icons/bluegithub.pdf_tex}{0.3em}}
\def\trademark{™}

\makeatletter
\DeclareRobustCommand\tlabel{%
  \tikz[overlay, remember picture, baseline=(nlabel.base)]{\node[anchor=base, inner sep=0pt] (nlabel) {};}%
}
\DeclareRobustCommand\posttlabel{%
  \tikz[overlay, remember picture]{
    \coordinate (ll) at (nlabel.south -| current page.north west);
    \path[fill=link] ($(ll)-(0.1cm,0pt)$) rectangle ++(0.2cm, 0.9em);
  }%
}
\renewcommand\section[1]{%
  \@startsection{section}{1}{\z@}{0.1\baselineskip}{0.1\baselineskip \@minus 0.1\baselineskip}{\bfseries\MakeUppercase}*{\tlabel{}#1\makebox[0pt]{\posttlabel}}%
}
\makeatother

\def\dashdiv{{%
  \def\sp{0.3em minus 0.1em}%
  \hspace{\sp}%
  \color{black!50!white}%
  --%
  \hspace{\sp}%
}}

\setlength{\parindent}{0em}
\setlength{\parskip}{1em plus 0.5em minus 0.5em}

\setitemize{topsep=0em, itemsep=0.5\parskip, partopsep=0em, parsep=0.5\parskip}

\colorlet{link}{blue!50!black}

\makeatletter
\DTMnewdatestyle{my}{%
  \renewcommand{\DTMDisplaydate}[4]{\DTMshortmonthname{##2} ##1}%
  \renewcommand{\DTMdisplaydate}{\DTMDisplaydate}%
}
\DTMsetdatestyle{my}
\usepackage[hidelinks, pdftex, pdfauthor={Leo Yan}, pdftitle={Leo Yan - \today}]{hyperref}
\makeatother

\pagestyle{empty}%


\def\header{%inkscape -D -z --file=bluegithub.svg --export-pdf=bluegithub.pdf --export-latex
  \DTMsetdatestyle{my}%
  \begin{tikzpicture}[every node/.style={inner ysep=0.1cm, inner xsep=0cm, anchor=north west}]
    \coordinate (topleft) at (0, 0);
    \coordinate (topright) at (\textwidth, 0);

    \node[font=\Large] (name) at (topleft) {Leo Yan};
    \node (bio) at (name.south west) {{\color{black!50!white}\emph{(\today)}} First year Electrical Engineering undergraduate at\ucsbicon{}\ucsbwordicon{}};

    \node[anchor=north east, text width=0.4\textwidth, align=flush right, inner ysep=0cm] (contacts) at (topright)
      {Leo Yan \\
        {\color{link}\href{https://github.com/RustColeone/}{\githubicon{} RustColeone}}%
          \IfFileExists{phone_number.txt}{\\ \input{phone_number.txt}}{}
        };

    \coordinate (bottomleft) at ($(bio.south west)+(0, -0.1cm)$);
    \coordinate (bottomright) at ($(bottomleft)+(\textwidth, 0)$);

    \path[use as bounding box] (topleft) rectangle (bottomright);

    \path[draw] ($(bottomleft)+(-0.2cm,0)$) -- ($(bottomright)+(0.2cm,0)$);
  \end{tikzpicture}%
}

\def\ghurl#1{%
    \href{#1}{({\blueghicon{}})}%
  }

\begin{document}
  \header

  \section{Education}

  \begin{itemize}
    \def\ongoing{%
      \tikz[baseline=(t.base)]{
        \node[fill=link!10!white, use as bounding box, inner sep=0.25em, font=\small] (t) {\textsc{1st year}}
      }%
    }
    \item \ongoing{} \textbf{BS Electrical Engineering} \dashdiv{} University of California Santa Barbara \dashdiv{} 2019-2023

      Relevant modules \& coursework (able to provide code upon request):

      \begin{itemize}

        \item Intro to ECE (C++)

        \begin{itemize}
          \item Using Arduino for problem solving and project buildings in C++.

          \begin{itemize}
            \item Digital to analog converter, using R2R Ladder
            \item Experimented with different components and sensors
            \item Built individual project \dashdiv{} \href{https://github.com/RustColeone/OLEDGameConsole}{\color{link}\textbf{Arduino Gaming Console}} \ghurl{https://github.com/RustColeone/OLEDGameConsole}
          \end{itemize}
        \end{itemize}

        \item Signal Processing (Python)

        \begin{itemize}
          \item Learned basics to Signal Processing and applied knowledge with Python.
          \begin{itemize}
            \item Applied Fourier Transform to different signals to analyze them.
            \item FIR Filters to smoothen images in lab.
            \item Shannon's theories.
          \end{itemize}
        \end{itemize}

        \item Computer Science (C++)

        \begin{itemize}
          \item Programming guidelines and basics
          \item Modern tools for software developments in C++
        \end{itemize}
      \end{itemize}

    \item \textbf{A-Level} \dashdiv{} (Shenzhen, China) Nanshan Chinese International College \dashdiv{} 2017-2019

      Further Math, Math, Physics and Chemistry \dashdiv{} A*/A /A*/A /
  \end{itemize}

  \section{Skills}

  \begin{itemize}[itemsep=0.1\parskip]
    \item Languages: Java, C++, C\#, Python, Verilog, \LaTeX

    \begin{itemize}
      \item Currently learning C++ and Python
      \item Has experience developing in Quartus
      \item This documents and the notes I took in class are written in \LaTeX
    \end{itemize}

    \item Circuit Board Designs
    \begin{itemize}
      \item Familiar with developing Circuit Boards with PADs
    \end{itemize}

    \item Web development
    \begin{itemize}
      \item Worked with web developers in Tencent as an intern.
    \end{itemize}
    
    \item Game development
    \begin{itemize}
      \item Building and designing games with Unity
    \end{itemize}

  \end{itemize}

  \section{Personal Projects}

  \begin{itemize}

    \item \href{https://github.com/RustColeone/ArduinoDebugger}{\color{link}\textbf{Arduino Debugger}} \ghurl{https://github.com/RustColeone/ArduinoDebugger}: Single Step Arduino Debugging made cheap and affordable \dashdiv{} 2018 to date.

    \begin{itemize}
      \item PC and Arduino communicate through serial communication
      \item HID, bluetooth and WiFi to be supported in the future
      \item PC side runs library identical to Arduino but sents an command instead of executing
      \item Arduino is programmed to translate the command into corresponding code and execute it
    \end{itemize}

    \item \href{https://github.com/RustColeone/ArduinoRemoteDisplay}{\color{link}\textbf{Arduino Remote Display}} \ghurl{https://github.com/RustColeone/ArduinoRemoteDisplay}: Remotelly update display with MQTT protocol\dashdiv{} Oct 2019 to date

    \begin{itemize}
      \item Translates an image file with C\# desktop application and sent using MQTT
      \item Arduino receives and output the images to the epaper display
      \item More display to be supported in the future
      \item A test for adding wireless step by step debug function to Arduino Debugger
    \end{itemize}

    \item \href{https://github.com/RustColeone/GameBoyPi}{\color{link}\textbf{GameBoy-Pi}} \ghurl{https://github.com/RustColeone/GameBoyPi}: Gameboy built with Arduino \dashdiv{} 2018

    \begin{itemize}
      \item Circuit designed by me and manufactured by \href{https://jlcpcb.com/}{\color{link}JLCPCB}
      \item Based on a Raspberry Pi 3B running \href{https://retropie.org.uk/}{\color{link}Retro Pie}
      \item Bought the gameboy case, screen and other components online
    \end{itemize}

    \item \href{https://github.com/RustColeone/PiPortable}{\color{link}\textbf{Arduino Calculator}} \ghurl{https://github.com/RustColeone/PiPortable}: Arduino Calculator(Pi Portable)\dashdiv{} 2017

    \begin{itemize}
      \item Portable Linux CLI, so I can program and have fun on the go.
      \item The Arduino communicates with the Rasbberry Pi zero, and acts as a keyboard and display.
      \item The display copy and outputs the CLI from the PI Zero.
    \end{itemize}

    \item \href{https://github.com/RustColeone/LEDEmojiGoggle}{\color{link}\textbf{LED Emoji Goggle}} \ghurl{https://github.com/RustColeone/LEDEmojiGoggle}: Inspired by Wrench (Watch Dogs 2)\dashdiv{} 2017-2018

    \begin{itemize}
      \item An goggle with LED Matrix built on to it
      \item LED controlled by CMOS 164 shift register IC
      \item Remotelly controlled with HC-05 bluetooth module and Arduino
      \item \href{https://www.bilibili.com/video/av33283411/}{\color{link}Video Demonstration at \bilibiliicon{}}
    \end{itemize}

  \end{itemize}

  \section{Other Interests}

  \begin{itemize}[itemsep=0.1\parskip]
    \item Interested in animation and some arts; 
    \item Been playing Piano for a long time;
    \item Video Editing
    \begin{itemize}
      \item \href{https://www.bilibili.com/video/av55212868/}{\color{link}High School Graduation Video \bilibiliicon{}};
      \item Sharing the process of how I make my projects.
      \item Videos to teach my audience something interesting.
      \item The music I written will also be posted.
      \item All my videos are uploaded to \href{https://www.bilibili.com/}{\bilibiliicon{}\bilibiliwordicon{}}
    \end{itemize}
  \end{itemize}
  \mbox{}
  \vfill{
    \begin{tikzpicture}
      \path[draw] ($(bottomleft)+(-0.2cm,0)$) -- ($(bottomright)+(0.2cm,0)$);
    \end{tikzpicture}%
    \begin{flushright}
      \begin{tabular}{ c c }
        \bilishield{} & \gitshield{}
      \end{tabular}
    \end{flushright}
  }
\end{document}
